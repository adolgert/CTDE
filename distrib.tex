\documentclass{article}
\usepackage{listings}

\title{Working Distributions}
\author{Drew Dolgert}
\date{\today}

\newcommand{\code}[1]{\texttt{#1}}

\begin{document}
\maketitle

\section{Shifted Absolute Distributions}
Our mission is to express probability distributions
with respect to some enabling time $t_e$ and then, after
some later time $t_0$, to perform three operations, sampling,
integration of the hazard, and the inverse of integration of
the hazard.

Given a cumulative distribution,we normally write
\begin{equation}
  F(t)=1-e^{-\int_{0}^t \lambda(s)ds}\label{eqn:simplecdf}
\end{equation}
or
\begin{equation}
  F(t)=\int_0^t f(s)ds,
\end{equation}
but for distributions in absolute time, meaning they are shifted
by $t_e$, it is
\begin{equation}
  F(t,t_e)=1-e^{-\int_{0}^{t-t_e} \lambda(s)ds}
\end{equation}
or
\begin{equation}
  F(t,t_e)=\int_{0}^{t-t_e} f(s)ds.
\end{equation}
The call for $F(t)$ is \code{cdf(t)}, and $F(t,t_e)$ is \code{cdf($t-t_e$)}.

Now let's sample the distribution after some time $t_0>=t_e$, at which time
we normalize the distribution to its remaining probability. We can
think of this best with survivals, $G(t)=1-F(t)$. In words, the probability of
survival from $t_e$ to time $t$ ($G(t,t_e)$) is the probability
of survival from $t_e$ to $t_0$ ($G(t_0,t_e$)
\emph{and} the probability of survival from $t_0$ to $t$ ($G(t,_0,t_e)$).
\begin{equation}
  G(t,t_e)=G(t_0,t_e)G(t,t_0,t_e)
\end{equation}
Written in terms of hazards, this is
\begin{equation}
  e^{-\int_{0}^{t-t_e} \lambda(s)ds}=e^{-\int_{0}^{t_0-t_e} \lambda(s)ds}
    e^{-\int_{t_0-t_e}^{t-t_e} \lambda(s)ds},
\end{equation}
where the hazard is the same zero-based hazard from Eq.~\ref{eqn:simplecdf}.
Therefore, given the initial survival, expressed since the enabling time $t_e$,
the scaled survival is
\begin{equation}
  G(t,t_0,t_e)=G(t,t_e)/G(t_0,t_e).
\end{equation}
In terms of cumulative distribution functions,
\begin{eqnarray}
  F(t,t_0,t_e)&=&1-\frac{1-F(t,t_e)}{1-F(t_0,t_e)} \\
          &=&\frac{F(t,t_e)-F(t_0,t_e)}{1-F(t_0,t_e)}\label{eqn:shiftcum} \\
\end{eqnarray}
In terms of the function calls, that means
\begin{lstlisting}
  cdf(t, t0, te)=(cdf(t-te)-cdf(t0-te))/(1-cdf(t0,te))
\end{lstlisting}

\begin{table}
\begin{tabular}{ll} \hline
\code{cdf(d,t)} & $F_d(t)$ \\
\code{quantile(d,q)} & $F_d^{-1}(q)$ \\
\code{logcdf(d,t)} & $\ln(F_d(t))$ \\
\code{ccdf(d,t)} & $G_d(t)$ \\
\code{logccdf(d,t)} & $-\int_0^t \lambda_d(s)ds$ \\
\code{quantile(d,q)} & $F_d^{-1}(q)$ \\
\code{cquantile(d,q)} & $F_d^{-1}(1-q)=G_d^{-1}(q)$ \\
\code{invlogcdf(d,lp)} & $F_d^{-1}(e^{l_p})$ \\
\code{invlogccdf(d,lp)} & $G_d^{-1}(e^{l_p})$ or $-\int_0^{t(l_p)}\lambda(s)ds=l_p$ \\\hline
\end{tabular}
\caption{Translation of methods into math.\label{fig:methodmath}}
\end{table}

In practice, sampling algorithms are specific to particular distributions.
They are formally equivalent to drawing a uniform random variable between
0 and 1, which we call $U$, and solving $U=F(t')$ for $t'$.
For the case of interest, where the distribution has an enabling time, $t_e$,
and is being observed after survival to a time $t_0$, sampling is formally
a solution $t'$ to $U=F(t', t_0, t_e)$. Looking back at Eq.~\ref{eqn:shiftcum},
we can write this as
\begin{eqnarray}
 U&=&F(t,t_0,t_e) \\
  &=&\frac{F(t,t_e)-F(t_0,t_e)}{1-F(t_0,t_e)} \\
U(1-F(t_0,t_e))&=&F(t,t_e)-F(t_0,t_e) \\
F(t,t_e)&=&U(1-F(t_0,t_e))+F(t_0,t_e) \\
F(t-t_e)&=&U(1-F(t_0-t_e))+F(t_0-t_e) \\
t-t_e &=& F^{-1}\left[U(1-F(t_0-t_e))+F(t_0-t_e)\right] \\
t &=& t_e+F^{-1}\left[U(1-F(t_0-t_e))+F(t_0-t_e)\right]
\end{eqnarray}
Using the inverse \textsc{CDF} from Table~\ref{fig:methodmath},
the inverse of this shifted quantile is
\begin{lstlisting}
  quantile(U, t0, te)=te+quantile(U+(1-U)*cdf(t0-te))
\end{lstlisting}
This would be a way to sample any distribution with a \textsc{CDF} and
quantile, but likely badly.

The next two pieces concern the hazard. The goal is to find the integral
of the hazard between two absolute times, $t_1$ and $t_2$, where both
are $t_{1,2}\ge t_0$. This is
\begin{equation}
  \int_{t_1-t_e}^{t_2-t_e} \lambda(s)ds=\int_{0}^{t_2-t_e} \lambda(s)ds
  	-\int_{0}^{t_1-t_e} \lambda(s)ds.
\end{equation}
In terms of the given methods, this would be, noting the minus sign
in the table,
\begin{lstlisting}
  hazard_int(t1, t2, te)=logccdf(t1-te)-logccdf(t2-te)
\end{lstlisting}

Last is the inverse hazard. We want to solve for $t'$ in
\begin{equation}
  x=\int_{t_0-t_e}^{t'-t_e}\lambda(s)ds.
\end{equation}
Expanding as before, this is
\begin{eqnarray}
  x&=&\int_{0}^{t'-t_e}\lambda(s)ds-\int_{0}^{t_0-t_e}\lambda(s)ds \\
  x+\int_{0}^{t_0-t_e}\lambda(s)ds&=&\int_{0}^{t'-t_e}\lambda(s)ds \\
  -x-\int_{0}^{t_0-t_e}\lambda(s)ds&=& -\int_{0}^{t'-t_e}\lambda(s)ds \\
  l_p&=&-x+\left[-\int_{0}^{t_0-t_e}\lambda(s)ds\right] \\
  l_p&=&-\int_{0}^{t'-t_e}\lambda(s)ds
\end{eqnarray}
Translating this into equations, we get
\begin{lstlisting}
  lp=-x+logccdf(t0-te)
  inv_hazard_int(x, t0, te)=te+invlogccdf(lp)
\end{lstlisting}

\section{Log-Logistic}
Working from wikipedia, because that's smart.
\begin{equation}
  F(x;\alpha, \beta)=\frac{1}{1+(x/\alpha)^{-\beta}}.
\end{equation}
We shift this to
\begin{equation}
  F(t, t_e)=\frac{1}{1+((t-t_e)/\alpha)^{-\beta}}.
\end{equation}
The pdf is
\begin{equation}
  f(x;\alpha, \beta)=\frac{(\beta/\alpha)(x/\alpha)^{\beta-1}}
  {(1+(x/\alpha)^\beta)^2}.
\end{equation}
The quantile is
\begin{equation}
  F^{-1}(p; \alpha, \beta)=\alpha \left(\frac{p}{1-p}\right)^{1/\beta}.
\end{equation}
Survival
\begin{equation}
  G(t)=1-F(t)=\frac{1}{1+(t/\alpha)^\beta}.
\end{equation}
Hazard
\begin{equation}
  \lambda(t)=\frac{f(t)}{G(t)}=\frac{(\beta/\alpha)(t/\alpha)^{\beta-1}}
  {1+(t/\alpha)^\beta}
\end{equation}
Lastly, we need \code{invlogccdf(d,lp)}, which is $G_d^{-1}(e^{l_p})$,
or $-\int_0^t(l_p)\lambda(s)ds=l_p$.
\begin{eqnarray}
  l_p&=&\ln(G(t)) \\
  e^{l_p}&=&G(t) \\
  e^{l_p}&=&\frac{1}{1+(t/\alpha)^\beta} \\
  e^{-l_p}&=&1+(t/\alpha)^\beta \\
  (t/\alpha)^\beta&=&  1-e^{-l_p}\\
  t/\alpha&=& (1-e^{-l_p})^{1/\beta}\\
   t&=&\alpha(1-e^{-l_p})^{1/\beta}\\
\end{eqnarray}


\end{document}
