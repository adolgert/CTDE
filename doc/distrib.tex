\documentclass{article}
\usepackage{listings}

\title{Working Distributions}
\author{Drew Dolgert}
\date{\today}

\newcommand{\code}[1]{\texttt{#1}}

\begin{document}
\maketitle

\section{Shifted Absolute Distributions}
Our mission is to express probability distributions
with respect to some enabling time $t_e$ and then, after
some later time $t_0$, to perform three operations, sampling,
integration of the hazard, and the inverse of integration of
the hazard.

Given a cumulative distribution,we normally write
\begin{equation}
  F(t)=1-e^{-\int_{0}^t \lambda(s)ds}\label{eqn:simplecdf}
\end{equation}
or
\begin{equation}
  F(t)=\int_0^t f(s)ds,
\end{equation}
but for distributions in absolute time, meaning they are shifted
by $t_e$, it is
\begin{equation}
  F(t,t_e)=1-e^{-\int_{0}^{t-t_e} \lambda(s)ds}
\end{equation}
or
\begin{equation}
  F(t,t_e)=\int_{0}^{t-t_e} f(s)ds.
\end{equation}
The call for $F(t)$ is \code{cdf(t)}, and $F(t,t_e)$ is \code{cdf($t-t_e$)}.

Now let's sample the distribution after some time $t_0>=t_e$, at which time
we normalize the distribution to its remaining probability. We can
think of this best with survivals, $G(t)=1-F(t)$. In words, the probability of
survival from $t_e$ to time $t$ ($G(t,t_e)$) is the probability
of survival from $t_e$ to $t_0$ ($G(t_0,t_e$)
\emph{and} the probability of survival from $t_0$ to $t$ ($G(t,_0,t_e)$).
\begin{equation}
  G(t,t_e)=G(t_0,t_e)G(t,t_0,t_e)
\end{equation}
Written in terms of hazards, this is
\begin{equation}
  e^{-\int_{0}^{t-t_e} \lambda(s)ds}=e^{-\int_{0}^{t_0-t_e} \lambda(s)ds}
    e^{-\int_{t_0-t_e}^{t-t_e} \lambda(s)ds},
\end{equation}
where the hazard is the same zero-based hazard from Eq.~\ref{eqn:simplecdf}.
Therefore, given the initial survival, expressed since the enabling time $t_e$,
the scaled survival is
\begin{equation}
  G(t,t_0,t_e)=G(t,t_e)/G(t_0,t_e).
\end{equation}
In terms of cumulative distribution functions,
\begin{eqnarray}
  F(t,t_0,t_e)&=&1-\frac{1-F(t,t_e)}{1-F(t_0,t_e)} \\
          &=&\frac{F(t,t_e)-F(t_0,t_e)}{1-F(t_0,t_e)}\label{eqn:shiftcum} \\
\end{eqnarray}
In terms of the function calls, that means
\begin{lstlisting}
  cdf(t, t0, te)=(cdf(t-te)-cdf(t0-te))/(1-cdf(t0,te))
\end{lstlisting}

\begin{table}
\begin{tabular}{ll} \hline
\code{cdf(d,t)} & $F_d(t)$ \\
\code{quantile(d,q)} & $F_d^{-1}(q)$ \\
\code{logcdf(d,t)} & $\ln(F_d(t))$ \\
\code{ccdf(d,t)} & $G_d(t)$ \\
\code{logccdf(d,t)} & $-\int_0^t \lambda_d(s)ds$ \\
\code{quantile(d,q)} & $F_d^{-1}(q)$ \\
\code{cquantile(d,q)} & $F_d^{-1}(1-q)=G_d^{-1}(q)$ \\
\code{invlogcdf(d,lp)} & $F_d^{-1}(e^{l_p})$ \\
\code{invlogccdf(d,lp)} & $G_d^{-1}(e^{l_p})$ or $-\int_0^{t(l_p)}\lambda(s)ds=l_p$ \\\hline
\end{tabular}
\caption{Translation of methods into math.\label{fig:methodmath}}
\end{table}

In practice, sampling algorithms are specific to particular distributions.
They are formally equivalent to drawing a uniform random variable between
0 and 1, which we call $U$, and solving $U=F(t')$ for $t'$.
For the case of interest, where the distribution has an enabling time, $t_e$,
and is being observed after survival to a time $t_0$, sampling is formally
a solution $t'$ to $U=F(t', t_0, t_e)$. Looking back at Eq.~\ref{eqn:shiftcum},
we can write this as
\begin{eqnarray}
 U&=&F(t,t_0,t_e) \\
  &=&\frac{F(t,t_e)-F(t_0,t_e)}{1-F(t_0,t_e)} \\
U(1-F(t_0,t_e))&=&F(t,t_e)-F(t_0,t_e) \\
F(t,t_e)&=&U(1-F(t_0,t_e))+F(t_0,t_e) \\
F(t-t_e)&=&U(1-F(t_0-t_e))+F(t_0-t_e) \\
t-t_e &=& F^{-1}\left[U(1-F(t_0-t_e))+F(t_0-t_e)\right] \\
t &=& t_e+F^{-1}\left[U(1-F(t_0-t_e))+F(t_0-t_e)\right]
\end{eqnarray}
Using the inverse \textsc{CDF} from Table~\ref{fig:methodmath},
the inverse of this shifted quantile is
\begin{lstlisting}
  quantile(U, t0, te)=te+quantile(U+(1-U)*cdf(t0-te))
\end{lstlisting}
This would be a way to sample any distribution with a \textsc{CDF} and
quantile, but likely badly.

The next two pieces concern the hazard. The goal is to find the integral
of the hazard between two absolute times, $t_1$ and $t_2$, where both
are $t_{1,2}\ge t_0$. This is
\begin{equation}
  \int_{t_1-t_e}^{t_2-t_e} \lambda(s)ds=\int_{0}^{t_2-t_e} \lambda(s)ds
  	-\int_{0}^{t_1-t_e} \lambda(s)ds.
\end{equation}
In terms of the given methods, this would be, noting the minus sign
in the table,
\begin{lstlisting}
  hazard_int(t1, t2, te)=logccdf(t1-te)-logccdf(t2-te)
\end{lstlisting}

Last is the inverse hazard. We want to solve for $t'$ in
\begin{equation}
  x=\int_{t_0-t_e}^{t'-t_e}\lambda(s)ds.
\end{equation}
Expanding as before, this is
\begin{eqnarray}
  x&=&\int_{0}^{t'-t_e}\lambda(s)ds-\int_{0}^{t_0-t_e}\lambda(s)ds \\
  x+\int_{0}^{t_0-t_e}\lambda(s)ds&=&\int_{0}^{t'-t_e}\lambda(s)ds \\
  -x-\int_{0}^{t_0-t_e}\lambda(s)ds&=& -\int_{0}^{t'-t_e}\lambda(s)ds \\
  l_p&=&-x+\left[-\int_{0}^{t_0-t_e}\lambda(s)ds\right] \\
  l_p&=&-\int_{0}^{t'-t_e}\lambda(s)ds
\end{eqnarray}
Translating this into equations, we get
\begin{lstlisting}
  lp=-x+logccdf(t0-te)
  inv_hazard_int(x, t0, te)=te+invlogccdf(lp)
\end{lstlisting}

\section{Log-Logistic}
Working from wikipedia, because that's smart.
\begin{equation}
  F(x;\alpha, \beta)=\frac{1}{1+(x/\alpha)^{-\beta}}.
\end{equation}
We shift this to
\begin{equation}
  F(t, t_e)=\frac{1}{1+((t-t_e)/\alpha)^{-\beta}}.
\end{equation}
The pdf is
\begin{equation}
  f(x;\alpha, \beta)=\frac{(\beta/\alpha)(x/\alpha)^{\beta-1}}
  {(1+(x/\alpha)^\beta)^2}.
\end{equation}
The quantile is
\begin{equation}
  F^{-1}(p; \alpha, \beta)=\alpha \left(\frac{p}{1-p}\right)^{1/\beta}.
\end{equation}
Survival
\begin{equation}
  G(t)=1-F(t)=\frac{1}{1+(t/\alpha)^\beta}.
\end{equation}
Hazard
\begin{equation}
  \lambda(t)=\frac{f(t)}{G(t)}=\frac{(\beta/\alpha)(t/\alpha)^{\beta-1}}
  {1+(t/\alpha)^\beta}
\end{equation}
Lastly, we need \code{invlogccdf(d,lp)}, which is $G_d^{-1}(e^{l_p})$,
or $-\int_0^t(l_p)\lambda(s)ds=l_p$.
\begin{eqnarray}
  l_p&=&\ln(G(t)) \\
  e^{l_p}&=&G(t) \\
  e^{l_p}&=&\frac{1}{1+(t/\alpha)^\beta} \\
  e^{-l_p}&=&1+(t/\alpha)^\beta \\
  (t/\alpha)^\beta&=&  1-e^{-l_p}\\
  t/\alpha&=& (1-e^{-l_p})^{1/\beta}\\
   t&=&\alpha(1-e^{-l_p})^{1/\beta}\\
\end{eqnarray}

\section{Gamma}
We will define paramaters from the shape $\alpha$ and rate $\beta$.
\begin{equation}
  f(x)=\frac{\beta^\alpha}{\Gamma(\alpha)}x^{\alpha-1}e^{-\beta x}
\end{equation}
where
\begin{equation}
  \Gamma(t)=\int_0^\infty x^{t-1}e^{-x}dx.
\end{equation}
The CDF is
\begin{equation}
  F(x;\alpha,\beta)=\frac{\gamma(\alpha,\beta x)}{\Gamma(\alpha)}
\end{equation}
where $\gamma$ is the (lower) incomplete gamma function,
\begin{equation}
  \gamma(x;\alpha)=\int_0^x t^{\alpha-1}e^{-t}dt
\end{equation}

In our back pocket, from \texttt{Boost::Math}, are
$\Gamma(x)$, $\ln(|\Gamma(x)|)$, digamma, which is
\begin{equation}
  \psi(x)=\frac{d}{dx}\ln(\Gamma(x))=\frac{\Gamma'(x)}{\Gamma(x)},
\end{equation}
gamma ratio, which is $\Gamma(a)/\Gamma(b)$,
gamma delta ratio, which is $\Gamma(a)/\Gamma(a+\Delta)$,
and the set of incomplete gamma functions.
In order, they are normalized lower incomplete, normalized upper, incomplete
full (non-normalized) lower incomplete, and full (non-normalized)
upper incomplete gamma functions.
\begin{eqnarray}
  \mbox{gamma\_p}(a,z)&=&\frac{\gamma(a,z)}{\Gamma(a)}=\frac{1}{\Gamma(a)}
     \int_0^zt^{a-1}e^{-t}dt \\
  \mbox{gamma\_q}(a,z)&=&\frac{\Gamma(a,z)}{\Gamma(a)}=\frac{1}{\Gamma(a)}
     \int_z^0t^{a-1}e^{-t}dt \\
  \mbox{tgamma\_lower}(a,z)&=&\gamma(a,z)=
     \int_0^zt^{a-1}e^{-t}dt \\
  \mbox{tgamma}(a,z)&=&\Gamma(a,z)=\frac{1}{\Gamma(a)}
     \int_z^0t^{a-1}e^{-t}dt \\
\end{eqnarray}
There are a set of inverses of incomplete gamma functions
and derivatives of incomplete gamma functions.
OK, back to what we need.
\begin{eqnarray}
  F(x;\alpha,\beta)&=&\mbox{gamma\_p}(\alpha, \beta x) \\
  F^{-1}(y;\alpha,\beta)&=&\mbox{gamma\_p\_inv}(\alpha, y)/\beta
\end{eqnarray}

The hazard integral, in terms of the cdf, is
\begin{eqnarray}
 \int_{t_1-t_e}^{t_2-t_e}\lambda(s)ds&=&-\ln(1-F(t_2-t_e))+\ln(1-F(t_1-t_e)) \\
 &=& \ln\left[\frac{1-F(t_1-t_e)}{1-F(t_2-t_e)}\right].
\end{eqnarray}
Can we simplify this into something provided?
\begin{eqnarray}
\int_{t_1-t_e}^{t_2-t_e}\lambda(s)ds & = & \ln\left[\frac{1-\frac{\gamma(\alpha,\beta (t_1-t_e))}{\Gamma(\alpha)}}{1-\frac{\gamma(\alpha,\beta (t_2-t_e))}{\Gamma(\alpha)}}\right] \\
 & = & \ln\left[\frac{\Gamma(\alpha)-\gamma(\alpha,\beta (t_1-t_e))}
 {\Gamma(\alpha)-\gamma(\alpha,\beta (t_2-t_e))} \right] \\
\gamma(\alpha,\beta (t_1-t_e)) & = & \int_0^{\beta(t_1-t_e)} t^{\alpha-1}e^{-t}dt
\end{eqnarray}
It looks like we might do best just with
\begin{lstlisting}
Ga=tgamma(a)
hazint(te, t1, t2)=log((Ga-tgamma_lower(a,b*(t1-te)))/
    (Ga-tgamma_lower(a,b*(t2-te))))
\end{lstlisting}

Our other goal for Gamma distributions is to get the inverse hazard.
This can be seen as two steps. First find the integral
\begin{equation}
  l_p=-x+\left[\int_0^{t0-t_e}\lambda(s)ds\right].
\end{equation}
Then solve for $t'$ in
\begin{equation}
  l_p=-\int_0^{t'-t_e}\lambda(s)ds.
\end{equation}
Or, we could write this as
\begin{equation}
  l_e =e^{-x}e^{-\int_0^{t0-t_e}\lambda(s)ds}=e^{-x}(1-F(t_0-t_e))
\end{equation}
and
\begin{equation}
  l_e=e^{-\int_0^{t'-t_e}\lambda(s)ds}=1-F(t'-t_e).
\end{equation}
All at once,
\begin{eqnarray}
  F(t'-t_e)&=&1-e^{-x}(1-F(t_0-t_e)) \\
 t'&=&t_e+F^{-1}\left(1-e^{-x}(1-F(t_0-t_e))\right). \\
 F(t_0-t_e)&=&\mbox{gamma\_p}(\alpha,\beta(t_0-t_e)) \\
 F^{-1}(y)&=&\mbox{gamma\_p\_inv}(\alpha, y)/\beta
\end{eqnarray}
So here is our inverse hazard integral.
\begin{lstlisting}
  quad=1-exp(-x)*(1-gamma_p(a,b*(t0-te)))
  tp=te + gamma_p_inv(a, quad)/b
\end{lstlisting}

\section{Uniform Distribution}
Maybe this one will be easier.
This distribution has two parameters, a start time
and an end time, $t_a$ and $t_b$.
The pdf is constant, $f(t)=1/(t_b-t_a)$ between
$t_a\le t<t_b$. The CDF is just the integral of
that, $F(t)=(t-t_a)/(t_b-t_a)$.
The integrated hazard will have nonzero cases for
for $t_1<t_a<t_2<t_b$, $t_1<t_a<t_b<t_2$,
$t_a<t_1<t_2<t_b$, $t_a<t_1<t_b<t_2$.
It is zero for $t_1<t_2<t_a$ and $t_a<t_b<t_1<t_2$
\begin{equation}
  \int_{t_1-t_e}^{t_2-t_e}\lambda(s)ds=
      \ln\left[\frac{1-F(t_1-t_e)}{1-F(t_2-t_e)}\right]
\end{equation}
If $t_a\le t_n-t_e<t_b$, then $F(t_n-t_e)=(t_n-t_e-t_a)/(t_b-t_a)$.
Otherwise it is $0$ or $1$. It should never be the
case that a uniform distribution does not fire
before $t_b$. The hazard integral always sums over
time already past in the simulation. Nevertheless, it will
be necessary to check for overflow near $t_b$, and it
would help to keep the two logs separated, instead of
in the fraction.

What about the inverse of the hazard integral?
$F^{-1}(x)=t_a+(t_b-t_a)x$ Therefore, for $t_a\le t_0-t_e$,
\begin{equation}
  t'=t_e+t_a+(t_b-t_a)\left[1-e^{-x}\left(1-\frac{t_0-t_e-t_a}{t_b-t_a}\right)\right]
\end{equation}
and for $t_0-t_e< t_a$,
\begin{equation}
  t'=t_e+t_a+(t_b-t_a)\left[1-e^{-x}\right]
\end{equation}

\section{Triangular Distribution}
The cumulative distribution function for the triangular distribution
with endpoints $a$ and $b$ and midpoint $m$ is
\begin{eqnarray}
  \frac{(t-a)^2}{(b-a)(m-a)} & & a\le t \le m \\
  1-\frac{(b-t)^2}{(b-a)(b-m)} & & m<t\le b.
\end{eqnarray}
This makes the survival
\begin{eqnarray}
  1-\frac{(t-a)^2}{(b-a)(m-a)} & & a\le t \le m \\
  \frac{(b-t)^2}{(b-a)(b-m)} & & m<t\le b.
\end{eqnarray}
\subsection{Simple Sample}
The cutoff is at $t=m$, which is
\begin{eqnarray}
  U'&=&\frac{(m-a)^2}{(b-a)(m-a)} \\
  &=&\frac{m-a}{b-a}
\end{eqnarray}
so first check whether $U$ is greater than that. Then, for $U$
less than that,
\begin{eqnarray}
  t = a + \left[U(b-a)(m-a)\right]^{1/2} & & U\le U' \\
  t = b- \left[(1-U)(b-a)(b-m)\right]^{1/2} & & U'<U \\
\end{eqnarray}

\subsection{Shifted Sample}
If this is sampled after some time, $x$, then the thing
we want to invert is
\begin{equation}
  U=\frac{F(t)-F(x)}{G(x)}
\end{equation}
so
\begin{equation}
  F(t)=UG(x)+F(x)
\end{equation}
which, for $a<t\le m$ and $a<x\le m$, is
\begin{eqnarray}
  \frac{(t-a)^2}{(b-a)(m-a)} & = & U\left[1-\frac{(x-a)^2}{(b-a)(m-a)}\right]
  +\frac{(x-a)^2}{(b-a)(m-a)} \\
  (t-a)^2 & = & U(b-a)(m-a)+ (1-U)(x-a)^2 \\
  t & = & a + \left[U(b-a)(m-a)+ (1-U)(x-a)^2\right]^{1/2}
\end{eqnarray}
For $m<t\le b$ and $m<x\le b$, this is
\begin{eqnarray}
  1-\frac{(b-t)^2}{(b-a)(b-m)} & = & U\frac{(b-x)^2}{(b-a)(b-m)}+
    1-\frac{(b-x)^2}{(b-a)(b-m)} \\
  -(b-t)^2 & = & U(b-x)^2-(b-x)^2 \\
  t & = & b-(b-x)\sqrt{1-U}
\end{eqnarray}
In the case that $m<t\le b$ and $a<x\le m$, the result is
\begin{eqnarray}
  1-\frac{(b-t)^2}{(b-a)(b-m)} & = & U\left[1-\frac{(x-a)^2}{(b-a)(m-a)}\right]
   + \frac{(x-a)^2}{(b-a)(m-a)}\\
  1-\frac{(b-t)^2}{(b-a)(b-m)} & = &U+\frac{(x-a)^2(1-U)}{(b-a)(m-a)} \\
  \frac{(b-t)^2}{(b-a)(b-m)} & = &(1-U)-\frac{(x-a)^2(1-U)}{(b-a)(m-a)} \\
  (b-t)^2& = &(1-U)(b-a)(b-m)-\frac{(x-a)^2(1-U)(b-m)}{(m-a)} \\
  t & = & b-\left[(1-U)(b-a)(b-m)-\frac{(x-a)^2(1-U)(b-m)}{(m-a)}\right]^{1/2}
\end{eqnarray}

\subsection{Sampling from Quantile}
The equation we have to solve is
\begin{equation}
  F(t)=1-G(t_j)(1-u)\prod_i\frac{G_i(t_i)}{G_i(t_{i+1})},
\end{equation}
given to us as
\begin{equation}
  F(t)=1-G(t_j)(1-u)\gamma,
\end{equation}
so, in terms of survivals, it's
\begin{equation}
  G(t)=G(t_j)(1-u)\gamma
\end{equation}
The value on the right is all known.
If $t_j<m$, the cutoff is
\begin{equation}
  G'=\frac{b-m}{(b-a)(b-m)}
\end{equation}
Below that,

\end{document}
